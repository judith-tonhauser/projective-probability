\documentclass{language}
\usepackage{lipsum}
\usepackage{times}
\usepackage{qtree}
\usepackage{gb4e}

\begin{document}

\title{Example Of Using The Class File}
\author{John Beavers\\Department of Linguistics\\The University of
  Texas at Austin\\305 E. 23rd Street, Mail Code B5100\\
Austin, TX 78712,
  USA
\and
 Noam Chomsky\\Department of Linguistics\\
P.O. Box 210025\\
The University of Arizona\\
Tucson, AZ 85721, USA}

\date{}

\maketitle

\begin{abstract}
\noindent\normalsize
This is how an abstract would be formatted.\ackfootnote{The
  acknowledgment footnote has a slightly different treatment.}

\keywords{Template, linguistics, Language}

\end{abstract}

\section{Introduction}

Here's a citation to \citet{Dowty79}, where we have parentheses, and
another one in \citealp{Chomsky57}, where we do not.\footnote{Here's a
  regular endnote. I'm testing whether if you make it longer it'll
  trigger the right skipping afterwards, because for some reason for a
  while we weren't getting a space between endnotes, but it looks good
  now.} Here's an example, using {\tt gb4e}:
%
\begin{exe}
  \ex\label{example}
  \begin{xlist}
    \ex This is an example.
    \ex This is another example.
    \end{xlist}
\end{exe}
%
Example \ref{example} is the only one I'll use, and the whole bit
about parentheses (e.g.\ \ref{example} vs. (\ref{example})) is
something users will have to do themselves since it'll depend on their
specific example package. The rest of this thing is mostly filled with
lorem text just to give it some length.

\subsection{Sub-introduction}
\fsubsubsection{Sub-sub-introduction}
\label{label1}

\lipsum

\begin{figure}
  \begin{center}
\Tree [.S [.NP D\\A N\\tree ] [.VP V\\is [.DP D\\a N\\figure ] ] ]
    \caption{A figure of some kind}
    
  \end{center}    
  
\end{figure}


\section{Background}

\lipsum

\begin{table}
  \begin{center}
    \begin{tabular}[t]{cccc}
      
      {\bf This} & {\bf is} & {\bf a} & {\bf table}\\\hline
      that & I & made & up
    \end{tabular}
    \caption{An example table}
    
  \end{center}    
  
\end{table}


\subsection{Formal framework}

\lipsum.\footnote{Another endnote.}

\begin{table}
  \begin{center}
    \begin{tabular}[t]{cccc}
      
      {\bf This} & {\bf is} & {\bf another} & {\bf table}\\\hline
      that & I & made & up
    \end{tabular}
    \caption{A second example table}
    
  \end{center}    
  
\end{table}


\section{Conclusion}
\fsubsection{Summary}

\lipsum



\appendix

\section{An Appendix}

If anyone wants one.

\listoftables

\listoffigures

%\bibliography{biblio}

\begin{thebibliography}{2}
\providecommand{\natexlab}[1]{#1}
\providecommand{\url}{\relax}
\providecommand{\urlprefix}{Online: }

\bibitem[{Chomsky(1957)}]{Chomsky57}
\textsc{Chomsky, Noam}. 1957.
\newblock \emph{Syntactic structures}.
\newblock The Hague, The Netherlands: Mounton.

\bibitem[{Dowty(1979)}]{Dowty79}
\textsc{Dowty, David}. 1979.
\newblock \emph{Word meaning and {M}ontague {G}rammar}.
\newblock Dordrecht: Reidel.

\end{thebibliography}


\fendnotes



\end{document}